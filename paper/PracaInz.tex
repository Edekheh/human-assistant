\documentclass{pginz}

%%%%%%%%%Miejsce na dodatkowe pakiety%%%%%%%%%%%%%
\usepackage{subcaption}

\begin{document}

%\includepdf[page={1}]{StronaTytulowa.pdf}
%\includepdf[page={1}]{Oswiadczenie.pdf}
\setcounter{page}{3}

\chapter*{Streszczenie}

Lorem ipsum dolor sit amet, consectetuer adipiscing elit, sed diam nonummy nibh euismod tincidunt ut laoreet dolore magna aliquam erat volutpat. Ut wisi enim ad minim veniam, quis nostrud exerci tation ullamcorper suscipit lobortis nisl ut aliquip ex ea commodo consequat. Duis autem vel eum iriure dolor in hendrerit in vulputate velit esse molestie consequat, vel illum dolore eu feugiat nulla facilisis at vero eros et accumsan et iusto odio dignissim qui blandit praesent luptatum zzril delenit augue duis dolore te feugait nulla facilisi. Nam liber tempor cum soluta nobis eleifend option congue nihil imperdiet doming id quod mazim placerat facer possim assum. Typi non habent claritatem insitam; est usus legentis in iis qui facit eorum claritatem. Investigationes demonstraverunt lectores legere me lius quod ii legunt saepius. Claritas est etiam processus dynamicus, qui sequitur mutationem consuetudium lectorum. Mirum est notare quam littera gothica, quam nunc putamus parum claram, anteposuerit litterarum formas humanitatis per seacula quarta decima et quinta decima. Eodem modo typi, qui nunc nobis videntur parum clari, fiant sollemnes in futurum.

\bigskip

\noindent\textbf{Słowa kluczowe:} lorem ipsum, sterowanie cyfrowe, ...

\bigskip

\noindent\textbf{Dziedzina nauki i techniki zgodna z OECD} Nauki inżynieryjne i techniczne, Elektrotechnika, elektronika i inżynieria informatyczna, Robotyka i Automatyka

\chapter*{Abstract}

Lorem ipsum dolor sit amet, consectetuer adipiscing elit, sed diam nonummy nibh euismod tincidunt ut laoreet dolore magna aliquam erat volutpat. Ut wisi enim ad minim veniam, quis nostrud exerci tation ullamcorper suscipit lobortis nisl ut aliquip ex ea commodo consequat. Duis autem vel eum iriure dolor in hendrerit in vulputate velit esse molestie consequat, vel illum dolore eu feugiat nulla facilisis at vero eros et accumsan et iusto odio dignissim qui blandit praesent luptatum zzril delenit augue duis dolore te feugait nulla facilisi. Nam liber tempor cum soluta nobis eleifend option congue nihil imperdiet doming id quod mazim placerat facer possim assum. Typi non habent claritatem insitam; est usus legentis in iis qui facit eorum claritatem. Investigationes demonstraverunt lectores legere me lius quod ii legunt saepius. Claritas est etiam processus dynamicus, qui sequitur mutationem consuetudium lectorum. Mirum est notare quam littera gothica, quam nunc putamus parum claram, anteposuerit litterarum formas humanitatis per seacula quarta decima et quinta decima. Eodem modo typi, qui nunc nobis videntur parum clari, fiant sollemnes in futurum.

\bigskip

\noindent\textbf{Keywords:} lorem ipsum, sterowanie cyfrowe, ...

\bigskip

\noindent\textbf{OECD consistent field of science and technology classification:} Nauki inżynieryjne i techniczne, Elektrotechnika, elektronika i inżynieria informatyczna, Robotyka i Automatyka


\tableofcontents
\addcontentsline{toc}{chapter}{Spis treści}

\chapter*{Lista symboli}

\begin{itemize}[noitemsep,topsep=0pt,parsep=0pt,partopsep=0pt,labelwidth=1cm,align=left,itemindent=0pt]
\item[$\mathbf{u}$] - wejście systemu
\item[$\mathbf{Q}_c$] - macierz kowariancji
\item[$F$] - napięcie $\left[ \frac{kg \cdot m}{s^2} \right]$ %\si[per-mode=fraction]{\kilo\gram\meter\per\second\squared}
\item[$R$] - rezystancja [\si{\ohm}]
\end{itemize}
\chapter*{Lista skrótów}

\begin{itemize}[noitemsep,topsep=0pt,parsep=0pt,partopsep=0pt,labelwidth=1cm,align=left,itemindent=0pt]
\item[SD] - Systemy Diagnostyki
\item[EKF] - Rozszerzony filtr Kalmana (ang. \textit{Extended Kalman Filter})
\end{itemize}

\chapter{Wstęp i cel pracy}


\noindent Od wielu lat robotyka prężnie się rozwija czego dowodem jest wzrost liczby publikacji dotyczących zagadnień ...

W robotyce można wyróżnić kilka podstawowych typów manipulatorów


Lorem ipsum dolor sit amet, consectetur adipiscing elit, sed do eiusmod tempor incididunt ut labore et dolore magna aliqua. Ut enim ad minim veniam, quis \textcolor{red}{nostrud exercitation $x=5y$ ullamco laboris nisi ut aliquip ex ea commodo consequat.} Duis aute irure dolor in reprehenderit in voluptate velit $$x=5y$$ esse cillum dolore eu fugiat nulla pariatur. \sout{\textbf{Excepteur sint occaecat cupidatat non proident, sunt in culpa} qui officia deserunt mollit anim id est laborum.}
\begin{equation}
\label{eq:prawo_newtona}
a = \geq \frac{F}{m}   
\end{equation}
Równanie nienumerowane
\begin{equation*}
\label{eq:prawo_newtona2}
\int\limits_{-\infty}^{6x+83^5} a_3^4 da_3 = \left( \frac{\mathbf{F}}{\sqrt{m^2}}    \right) \in \mathcal{C} \to \infty \begin{bmatrix}a   & b \\ c   & d  \end{bmatrix} \quad \qquad \text{for all $x$}
\end{equation*}


\noindent \textbf{Lista punktowana}

\begin{itemize}
    \item Punkt 1
    \item Punkt 2
    \begin{itemize}
        \item Podpunkt 2.1
        \item Podpunkt 2.2
        \begin{itemize}
            \item Podpunkt 2.2.1
            \item Podpunkt 2.2.2
        \end{itemize}
    \end{itemize}
    \item Punkt 3
\end{itemize}

\noindent \textbf{Lista numerowana}

\begin{enumerate}
    \item Punkt 1
    \item Punkt 2
    \begin{enumerate}
        \item Podpunkt 2.1
        \item Podpunkt 2.2
        \begin{itemize}
            \item Podpunkt 2.2.1
            \item Podpunkt 2.2.2
        \end{itemize}
    \end{enumerate}
    \item Punkt 3
\end{enumerate}

\section{Wstęp teoretyczny}

Podstawowym równaniem w teorii obwodów jest prawo Ohma dane następującym wzorem \cite{Now02}:

\begin{equation}
\label{eq:prawo_ohma}
U = I R    
\end{equation}
gdzie $U$ oznacza napięcie [\si{\volt}], $R$ oznacza rezystancję [\si{\ohm}], natomiast $I$ oznacza natężenie prądu [\si{\ampere}], patrz \cref{tab:nasza_tabela}. Kod pokazuje \cref{lst:kod}

\color{blue}
\subsection{Algorytmy szukania ścieżek}

\subsubsection{Algorytm A*}

co zostało uwzględnione w \cref{eq:prawo_ohma}, oraz użyte w \cite{Nat12,Bar04}

\subsubsection{Złożoność obliczeniowa} 


\section{Cel pracy}

Co widać na \cref{fig:nasz_obrazek}. \Cref{fig:nasz_obrazek} prezentuje

\begin{table}[b]%t - top, h - here, b - bottom
    \centering
    \caption{Podpis tabeli}
    \begin{tabular}{|c|c||}
        \hline
        \textbf{Nagłówek 1} & \textbf{Nagłówek 2 } \\ \hline
        4 & 89  \\ \hline
    \end{tabular}
    \label{tab:nasza_tabela}
\end{table}
\section{Układ pracy}

\cref{chap:przeglad} zawiera przegląd literatury, wyszczególniając ...

\begin{minipage}{\linewidth} % tego nie musi zawsze byc - dodale, zeby nie dzielilo kodu na dwie czesci
\begin{lstlisting}[language=Python,caption={Opis kodu},captionpos=b,label={lst:kod}]
import numpy as np
    
def incmatrix(genl1,genl2):
    m = len(genl1)
    n = len(genl2)
    M = None #to become the incidence matrix
    VT = np.zeros((n*m,1), int)  #dummy variable
    
    return M
\end{lstlisting}
\end{minipage}
\chapter[Przegląd literatury (Adam Nowak)]{PrzeglĄd literatury}
\label{chap:przeglad}
% tu będą kolejne rozdziały

\listoffigures
\addcontentsline{toc}{chapter}{Spis rysunków}
\listoftables
\addcontentsline{toc}{chapter}{Spis tabel}


%alternatywa - bibtex
\begin{thebibliography}{20}
\bibitem{Kow02} Kowalski J., Kabacki J.: Simulation of Network Systems in Education, Proceedings of the XXIV Autumn International Colloquium Advanced Simulation of Systems. ASIS 2002, 9-11 września 2002, Ostrava, Czechy, s. 213-218.
\bibitem{Nat12}	National Center of Biotechnology Information, http://www.ncbi.nlm.nih.gov (data dostępu 20.12.2012 r.).
\bibitem{Now02}	Nowak K.: Dydaktyczny model łączenia sieci LAN za pomocą sieci rozległych. Projekt dyplomowy inżynierski. WETI PG, 2002.
\bibitem{Bar04}	Barzykowski J. i inni: Współczesna metrologia – zagadnienia wybrane. WNT, Warszawa 2004, s. 575.
\end{thebibliography}

%*****************
%wymagane dodatki:
% Opis dyplomu
% Zawartość płyty CD
% Instrukcja dla projektanta
% Instrukcja dla użytkownika
\begin{appendices}
%\chapter[Instrukcja dla użytkownika]{Instrukcja dla u\.Zytkownika}
%\include{AppB}
\end{appendices}
%*****************

\end{document}